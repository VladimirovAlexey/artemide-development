%\documentclass[12pt]{article}
\documentclass[prd,nofootinbib,eqsecnum,final]{revtex4}
%,preprint,tightenlines,floatfix,showpacs,showkeys,preprintnumbers,
%\usepackage[dvips]{graphicx,color}
\usepackage{hyperref}
\usepackage{graphicx,color}
  \usepackage{bm}% bold math
   \usepackage{amsmath}
    \usepackage{amssymb}
     \usepackage{pifont}
%      \usepackage{simplewick}
%      \usepackage{srcltx}
\usepackage{tikz}
\usepackage[most]{tcolorbox}
\usepackage{rotating}
\usepackage{multirow}
\usepackage{longtable}
%\usepackage[makeroom]{cancel}
%\usepackage{fullpage}%full page style
\usepackage{listings}

%%%%%%%%%%%%%%%%%% ReNew Commands %%%%%%%%%%%%%%%%%%%%%%%%%%%%%%%
\newcommand{\Ds}{\displaystyle}

\newcommand{\nn}{\nonumber}

\newcommand{\tr}{\mathrm{tr}}
\newcommand{\Tr}{\mathrm{Tr}}
\newcommand{\sign}{\text{sign}}
\newcommand{\even}{\text{even}}
\newcommand{\odd}{\text{odd}}
\newcommand{\sh}{\text{sh}}
\newcommand{\ch}{\text{ch}}
\newcommand{\const}{\text{const.}}
\newcommand{\Li}{\text{Li}}
\newcommand{\ot}{\leftarrow}

\newcommand{\partialboth}{\!\!\stackrel{\leftrightarrow}{\partial}\!\!}

\renewcommand{\(}{\left(}
\renewcommand{\)}{\right)}
\renewcommand{\[}{\left[}
\renewcommand{\]}{\right]}

\renewcommand{\Im}{\text{Im}}
\renewcommand{\Re}{\text{Re}}

\renewcommand{\vec}[1]{\bm{#1}}
\newcommand{\fnot}[1]{\not{\! #1}}

%\definecolor{green}{rgb}{0.133,0.56,0}
\newcommand{\red}[1]{{\color[rgb]{1,0,0} #1}}
\newcommand{\blue}[1]{{\color{blue} #1}}
\newcommand{\gray}[1]{{\color{gray} #1}}

\newcommand{\bboxed}[1]{\blue{\boxed{#1}}}

\lstdefinestyle{DOS}
{
    backgroundcolor=\color{black},
    basicstyle=\scriptsize\color{white}\ttfamily
}
%%%%%%%%%%%%%%%%%%%%%CODE FROM INTERNET FOR GRID WITH COORDIATES%%%%
\makeatletter
\def\grd@save@target#1{%
  \def\grd@target{#1}}
\def\grd@save@start#1{%
  \def\grd@start{#1}}
\tikzset{
  grid with coordinates/.style={
    to path={%
      \pgfextra{%
        \edef\grd@@target{(\tikztotarget)}%
        \tikz@scan@one@point\grd@save@target\grd@@target\relax
        \edef\grd@@start{(\tikztostart)}%
        \tikz@scan@one@point\grd@save@start\grd@@start\relax
        \draw[minor help lines] (\tikztostart) grid (\tikztotarget);
        \draw[major help lines] (\tikztostart) grid (\tikztotarget);
        \grd@start
        \pgfmathsetmacro{\grd@xa}{\the\pgf@x/1cm}
        \pgfmathsetmacro{\grd@ya}{\the\pgf@y/1cm}
        \grd@target
        \pgfmathsetmacro{\grd@xb}{\the\pgf@x/1cm}
        \pgfmathsetmacro{\grd@yb}{\the\pgf@y/1cm}
        \pgfmathsetmacro{\grd@xc}{\grd@xa + \pgfkeysvalueof{/tikz/grid with coordinates/major step}}
        \pgfmathsetmacro{\grd@yc}{\grd@ya + \pgfkeysvalueof{/tikz/grid with coordinates/major step}}
        \foreach \x in {\grd@xa,\grd@xc,...,\grd@xb}
        \node[anchor=north] at (\x,\grd@ya) {\pgfmathprintnumber{\x}};
        \foreach \y in {\grd@ya,\grd@yc,...,\grd@yb}
        \node[anchor=east] at (\grd@xa,\y) {\pgfmathprintnumber{\y}};
      }
    }
  },
  minor help lines/.style={
    help lines,
    step=\pgfkeysvalueof{/tikz/grid with coordinates/minor step}
  },
  major help lines/.style={
    help lines,
    line width=\pgfkeysvalueof{/tikz/grid with coordinates/major line width},
    step=\pgfkeysvalueof{/tikz/grid with coordinates/major step}
  },
  grid with coordinates/.cd,
  minor step/.initial=.2,
  major step/.initial=1,
  major line width/.initial=1pt,
}

%%%%%%%%%%%%%%%%%%%%%%%%%%%%%%%%%%%%%%%%%%%%%%%%%%%%%%%%%%%%%%%%%%

\begin{document}
\title{Notes on the algorithms\\ used in artemide ver.3.}
\author{Alexey A. Vladimirov \\ \today}
\noaffiliation

\maketitle

\section{Convolution and grids for distributions with twist-2 input}

The convolution has the general form
\begin{eqnarray}
[C\otimes f](x,b,\mu)=\sum_f\int_x^1 \frac{dy}{y} C_{f,f'}\(\frac{x}{y},b,\mu\)f_{f'}(y,\mu).
\end{eqnarray}
Here the $\mu$ is internal parameter specified in the model. Furthermore, In the grid I save the function multiplied by $x$ I get
\begin{eqnarray}
x[C\otimes f](x,b,\mu)&=&
\sum_f\int_x^1 dy\frac{x}{y} C_{f,f'}\(\frac{x}{y},b,\mu\)f_{f'}(y,\mu)
\\\nn
&=&
\sum_f\int_x^1 dy\frac{x}{y} C_{f,f'}(y,b,\mu)f_{f'}\(\frac{x}{y},\mu\)
\\\nn
&=&
\sum_f\int_x^1 dy C_{f,f'}(y,b,)F_{f'}\(\frac{x}{y},\mu\),
\end{eqnarray}
where $F(x)=xf(x)$ is the function stored in LHAPDF grids.

\textbf{One needs a grid for this convolution in (x,b)}.

For it I use the algorithm from QCDnum. The coefficient function is presented as
\begin{eqnarray}
C_{f,f'}(x,b,\mu)=\sum_{n=0}\sum_{k=0}^n a_s^n \mathbf{L}^k C^{(n,k)}_{f,f'}(x).
\end{eqnarray}
The PDF is interpolated over the logarithmical grid
\begin{eqnarray}
\text{X-grid}:~~~[B_x,N_x],~~~x_i=10^{-B_x+\Delta_x i},~\text{with}~\Delta_x=\frac{B_x}{N_x},~i=0,...,N_x.
\end{eqnarray}
Such, grid span logarithmically $x\in[10^{-B},1]$. Over this grid I use qubic Lagrange interpolation
\begin{eqnarray}
F(x)=\sum_{i=0}^{N_x-1}\theta(x_i<x<x_{i+1})\sum_{r=-1,0,1,2} P_{i,r}(x)F(x_{i+r}),
\end{eqnarray}
where $P$ is the Lagrange polynomial over the logarithmic grid
\begin{eqnarray}
P_{i,r}(x)
=\prod_{\substack{l=-1,0,1,2\\ l\neq r}}\frac{\log_{10} x-\log_{10} x_{i+l}}{\log_{10} x_{i+k}-\log_{10} x_{i+l}}
=\prod_{\substack{l=-1,0,1,2\\ l\neq r}}\(\frac{\log_{10} \frac{x}{x_{i+k}}}{\Delta(k-l)}+1\),
\end{eqnarray}
The end segments are approximated by square polynomials, i.e. if $i=0$ $l\in{0,1,2}$ [+if $k<0$ $\to0$] if $i=N_x-1$ $l\in{-1,0,1}$ [+if $k>1$ $\to0$].

Now the function $f$ can be represented as
\begin{eqnarray}
F(x)=\sum_{i=0}^{N_x}F(x_i)W_i(x),
\end{eqnarray}
where 
\begin{eqnarray}
W_i(x)=\sum_{k=-2,-1,0,1}\theta(x_{i+k}<x<x_{i+k+1})P_{i+k,k}(x).
\end{eqnarray}
\begin{itemize}
\item the function $W$ has finite support $x_{i-2}<x<x_{i+2}$ and looks like a symmetric bump , with max at $x=x_i$
\item For $x=x_i$ (node) one has $W_i(x_j)=\delta_{ij}$
\end{itemize}
Therefore, the convolution can be presented as
\begin{eqnarray}
[C\otimes f](x,b,\mu)=\sum_{f'} \sum_{n=0}\sum_{k=0}^n \sum_{i=0}^{N_x-1} a_s^n \mathbf{L}^k F_{f'}(x_i,\mu) 
\mathfrak{T}_{ff'}^{(n,k)}(x,x_i),
\end{eqnarray}
where
\begin{eqnarray}
\mathfrak{T}_{ff'}^{(n,k)}(x,x_i)=\int_x^1 dy C^{(n,k)}_{ff'}(y)W_i\(\frac{x}{y}\).
\end{eqnarray}
For the $x=x_j \in $grid one has
\begin{eqnarray}
[C\otimes f](x_i,b,\mu)=\sum_{f'} \sum_{n=0}\sum_{k=0}^n \sum_{j=0}^{N_x-1} a_s^n \mathbf{L}^k  \mathfrak{T}_{ff';ij}^{(n,k)} f_{f'}(x_j,\mu) 
,
\end{eqnarray}
where
\begin{eqnarray}
\mathfrak{T}_{ff',ij}^{(n,k)}=\int_{x_i}^1 dy~C^{(n,k)}_{ff'}(y)W_j\(\frac{x_i}{y}\).
\end{eqnarray}
Note that since the function $W$ has restrictired support one has
\begin{eqnarray}
\mathfrak{T}_{ff',ij}^{(n,k)}=
\int^{\text{min}(1,x_i/x_{j-2})}_{\text{max}(x_i/x_{j+2},x_i)} dy C^{(n,k)}_{ff'}(y)W_j\(\frac{x_i}{y}\).
\end{eqnarray}
The matrix has the following form
\begin{eqnarray}
\left(
\begin{array}{cccccccc}
\mathfrak{T}_{00} & \mathfrak{T}_{01} & \mathfrak{T}_{02} & ... & \mathfrak{T}_{0(N-3)} & \mathfrak{T}_{0(N-2)} & \mathfrak{T}_{0(N-1)} & \mathfrak{T}_{0N}
\\
\red{\mathfrak{T}_{10}} & \red{\mathfrak{T}_{11}} & \red{\mathfrak{T}_{12}} & ... & \red{\mathfrak{T}_{1(N-3)}} & \mathfrak{T}_{1(N-2)} & \mathfrak{T}_{1(N-1)} & \mathfrak{T}_{1N}
\\
0 & \red{\mathfrak{T}_{10}} & \red{\mathfrak{T}_{11}} & ... & \red{\mathfrak{T}_{1(N-4)}} & \mathfrak{T}_{2(N-2)} & \mathfrak{T}_{2(N-1)} & \mathfrak{T}_{2N}
\\
0 & 0 & \red{\mathfrak{T}_{10}} & ... & \red{\mathfrak{T}_{1(N-5)}} & \mathfrak{T}_{3(N-2)} & \mathfrak{T}_{3(N-1)} & \mathfrak{T}_{3N}
\\ ... & ... &... &... &... &... &... &... 
\\
0 & 0 & 0 & ... & \red{\mathfrak{T}_{10}} & \mathfrak{T}_{(N-2)(N-2)} & \mathfrak{T}_{(N-2)(N-1)} & \mathfrak{T}_{(N-2)N}
\\
0 & 0 & 0 & ... & 0 & \mathfrak{T}_{(N-1)(N-2)} & \mathfrak{T}_{(N-1)(N-1)} & \mathfrak{T}_{(N-1)N}
\\
0 & 0 & 0 & ... & 0 & 0 & 0 & \mathfrak{T}_{NN}
\end{array}
\right),
\end{eqnarray}
where by red I show the part which has a pattern.

The coefficient function has three parts
\begin{eqnarray}
C(x)=C_\delta(x)+C_S(x)+C_R(x),
\end{eqnarray}
where
\begin{eqnarray}
C_\delta(x)=c_\delta \delta(1-x),\qquad
C_S(x)=c_S [g(x)]_+,
\end{eqnarray}
and $C_R$ is a regular part. Clearly the contribution of $C_\delta$ to $\mathfrak{T}$ is
\begin{eqnarray}
c_\delta \delta_{ij} \to \mathfrak{T}_{ff',ij}^{(n,k)}
\end{eqnarray}
The regular part contributes
\begin{eqnarray}
\int_{x_i/x_{j+2}}^{\min(1,x_i/x_{j-2})}dy C_R(y)W_j\(\frac{x_i}{y}\)\to \mathfrak{T}_{ff',ij}^{(n,k)}
\end{eqnarray}
Finally, the plus-part contributes
\begin{eqnarray}
&&\int_{x_i/x_{j+2}}^{\min(1,x_i/x_{j-2})}dy C_S(y)W_j\(\frac{x_i}{y}\)
=\int_{x_i/x_{j+2}}^{\min(1,x_i/x_{j-2})}dy c_S[g(y)]_+W_j\(\frac{x_i}{y}\)
\\\nn
&&=\int_{x_i/x_{j+2}}^{\min(1,x_i/x_{j-2})}dy c_S g(y)_+(W_j\(\frac{x_i}{y}\)-W_j(x_i))-c_S\int_0^{x_i/x_{j+2}} dy g(y)W_j(x_i)
\\\nn
&&=\int_{x_i/x_{j+2}}^{\min(1,x_i/x_{j-2})}dy c_S g(y)_+(W_j\(\frac{x_i}{y}\)-\delta_{ij})-c_S\int_0^{x_i/x_{j+2}} dy g(y)\delta_{ij}
\to \mathfrak{T}_{ff',ij}^{(n,k)},
\end{eqnarray}
where it is used that $W_j(x_i)=\delta_{ij}$.
\end{document}